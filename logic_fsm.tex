% Author: Dominic van der Zypen
% Editor: vi
% Last modified 2021-03-19

\documentclass[12pt]{amsart}
\usepackage{tikz}
\usetikzlibrary{arrows,shapes.gates.logic.US,shapes.gates.logic.IEC,calc}
\tikzstyle{branch}=[fill,shape=circle,minimum size=3pt,inner sep=0pt]
%\usepackage{amssymb}
%\usepackage{amsthm}
%\usepackage{amsfonts}
\newtheorem{lemma}{\bf Lemma}[section]
\newtheorem{theorem}[lemma]{\bf Theorem}
\newtheorem{corollary}[lemma]{\bf Corollary}
\newtheorem{proposition}[lemma]{\bf Proposition}
\newtheorem{fact}[lemma]{\bf Fact}
\newtheorem{definition}[lemma]{\bf Definition}
\newtheorem{remark}[lemma]{\bf Remark}
\newcommand{\restrict}{\mbox{$\mid$}}

\begin{document}

%...... title
\title{Converting Boolean functions to finite state machines}

%...... authors
\author{Dominic van der Zypen}
%\address{...Switzerland}
%\email{dominic.zypen@gmail.com}

%......MSC subject class
%\subjclass[2010]{05C15, 05C83}

%....... main()
\maketitle
\parindent = 0mm
\parskip = 2 mm
%--------------------------
\section{Problem setting}
We want to detect some events based on simple
Yes/No rules that are connected via {\tt AND} and
{\tt OR} gates.

The goal is to try to classify events into 
problematic vs non-problematic even if only partial
information is available.

Ficticious example:

Suppose we know that an event is only problematic if
\begin{itemize}
    \item the message transmitted contains the substring 
        {\tt ''execute''}, {\bf and}
    \item the port used is 8500.
\end{itemize}
Suppose we haven't analysed the message transmission yet, 
but we know that the message came through port 4112.
Then by the above AND-rule, we can already classify
as non-problematic and don't need to wait for the result
of the analysis of the message.

%--------------------------
\section{Disjuncitve Normal Form DNF}
Note that $\land$ denotes Boolean {\tt AND} (mnemonic:
$\land$ looks like 'A' as in AND), and $\lor$ denotes Boolean {\tt OR}
(mnemonic: $\lor$ looks like 'V' in Latin VEL meaning
OR/AND which exactly captures the meaning of logical OR). 

Any Boolean expression can be written in the form 

$$(X_{1,1}\land\ldots\land X_{1,n_1}) \lor (Y_{2,1}\land\ldots\land Y_{2,n_2})\lor \ldots
\lor (X_{k,1}\land\ldots\land X_{k,n_k})$$

where every $X_{i,j}$ is either a basic variable or a
negation of a basic variable.

We call this canonical form the {\em disjunctive normal form (DNF)}.

%--------------------------
\section{Simplifying assumptions}
\begin{enumerate}
    \item We can get by using NO NEGATIONS as we formulate triggers 
        positively.
    \item We assume our filter formula is given in {\em disjunctive normal
        form} (DNF) as introduced above. Any Boolean formula can be 
        transformed
        into DNF, and it is easier to do so if the formula doesn't include
        negatives (see item 1). 
    \item Per item to be analyzed, every trigger gets only set
        once to 0 or 1, there are no corrections or repetitions.
\end{enumerate}

%--------------------------
\section{Finite state machine: definition}
Formally, a {\em finite state machine (FSM)} is a 5-tuple $(\Sigma, S, s_0,
\delta, F)$ where
\begin{itemize}
        \item $\Sigma\neq \emptyset$ is the {\em input alphabet},
        \item $S$ is the {\em set of states},
        \item $s_0\in S$ is the {\em initial state},
        \item $\delta:S\times \Sigma \to S$ is the {\em transition 
            function},
        \item $F\subseteq S$ is the set of {\em final states}. 
            For ``ever-running'' machines, we may have $F=\emptyset$. 
\end{itemize}
Recall that for any Boolean expression, we can convert it to a canonical
form, the disjunctive normal form (DNF).  So there is a big 
OR gate, and we have 
small AND gates, all at the same level, leading into the big
OR gate. This is expressed with the formula we already saw:

$$(X_{1,1}\land\ldots\land X_{1,n_1}) \lor (Y_{2,1}\land\ldots\land Y_{2,n_2})\lor \ldots
\lor (X_{k,1}\land\ldots\land X_{k,n_k})$$

The goal for the remainder of this note is to build a finite state
machine (FSM) that can evaluate a Boolean formula given in DNF
even with only partial information - if that information is sufficient
to determine the truth value of the whole formula.

%--------------------------
\section{Building a DNF FSM}\label{fsm1}

There are several ways to build a finite state machine that
adheres to the disjunctive normal form. We present one
variant that appears quite natural - whether it 
will fit the business purposes, remains to be seen.

First we want to depict the DNF (disjunctive normal form) situation
graphically. By $x_{i,j}$ we denote the inputs (triggers),
some of which may be set to TRUE, others to FALSE, and
about some others the status may not be known.

\vspace*{3mm}

\begin{tikzpicture}[label distance=2mm]

    % ---- input nodes
    \node (x11) at (0,0) {$x_{1,1}$};
    \node (x1ldots) at (1,0) {$\ldots$};
    \node (x1n) at (2,0) {$x_{1,n_1}$};

    \node (x21) at (4,0) {$x_{2,1}$};
    \node (x2ldots) at (5,0) {$\ldots$};
    \node (x2n) at (6,0) {$x_{2,n_2}$};

    \node (ldots) at (8,0) {$\ldots$};

    \node (xk1) at (10,0) {$x_{k,1}$};
    \node (xkldots) at (11,0) {$\ldots$};
    \node (xkn) at (12,0) {$x_{k,n_k}$};
    
    \node (row) at ($(x1ldots)+(2,-2)$) {\scriptsize{(row of $k$ 
    {\tt AND}-gates)}};
        % print "k AND-gates"
    \node (bigor) at ($(x2ldots)+(4.5,-4)$) {\scriptsize{({\tt OR}-gate taking input from the 
    {\tt AND}-gates)}};
        % print "OR-gate"
    \node (output) at ($(x2ldots)+(1,-5.5)$) {\scriptsize{(output)}};
        % print "Output"
    % ---- gates
    \node[and gate US, draw, logic gate inputs=nnn, rotate=-90] at ($(x11)+(1,-2)$) (And1) {};
    \node[and gate US, draw, logic gate inputs=nnn, rotate=-90] at ($(x21)+(1,-2)$) (And2) {};
    \node[and gate US, draw, logic gate inputs=nnn, rotate=-90] at ($(xk1)+(1,-2)$) (Andk) {};
    % -> leading into one big AND gate:
    \node[or gate US, draw, logic gate inputs=nnn, rotate=-90] at ($(x2ldots)+(1,-4)$) (Or0) {};

    % ---- connections
    
    % 1) OR gate 1
    \draw (x11)  -- ++(0,-1) -| (And1.input 3);
            % |-  and -| instead of -- for "rectangular" drawing
    \draw (x1ldots) |- (And1.input 2);
    \draw (x1n) -- ++(0,-1) -| (And1.input 1);

    % 1) OR gate 2
    \draw (x21)  -- ++(0,-1) -| (And2.input 3);
    \draw (x2ldots) |- (And2.input 2);
    \draw (x2n) -- ++(0,-1) -| (And2.input 1);

    % 3) OR gate k
    \draw (xk1)  -- ++(0,-1) -| (Andk.input 3);
    \draw (xkldots) |- (Andk.input 2);
    \draw (xkn) -- ++(0,-1) -| (Andk.input 1);
    
    % connections from the OR gates to the BIG AND gate:

    \draw (And1.output)  -- ++(0,-0.5) -| (Or0.input 3);
    \draw (And2.output)  -- ++(0,-0.2) -| (Or0.input 2);
    \draw (Andk.output)  -- ++(0,-0.5) -| (Or0.input 1);

    % connections from BIG AND gate to (output)

    \draw[->] (Or0.output)  -> ++(0,-0.8); % [->] = arrow

\end{tikzpicture}
\begin{lemma}\label{classify}
It is easy to classify the output given the following
kinds of partially available inputs:
\begin{enumerate}
    \item If for every {\tt AND}-gate, we have at least 
        one false input, the output is FAIL (false).
    \item If for at least one {\tt AND}-gate, say 
        ${\tt AND}_i$, where $i\in\{1,\ldots,k\}$, all the inputs 
        are TRUE, or formally,
        if $$x_{i,j} = 1 \text{ for all } j\in \{1,\ldots,n_i\}$$
        then the output is HIT (true).
    \item Otherwise, the output is undefined.
\end{enumerate}
\end{lemma}
Based on this easy classification, we will now give a mathematical
description adhering to the definition of a finite state machine.

As depicted above, we have 1 big {\tt OR}-gate, with
$k$ {\tt AND}-gates are feeding into it. The {\tt AND}-gate number $i$
has $n_i$ inputs which we call ``triggers''. Each trigger can 
be set to TRUE (T), FALSE (F), or it is undefined.

\subsection{Input alphabet $\Sigma$} We set  $\Sigma = \{1,\ldots,k\} \times \{T, F\}$. 
    That is, an ``input event'' consists of the number of the {\tt AND}-gate 
    into which the corresponding trigger feeds, and the truth value 
    indecating whether the trigger is set to T or F. So, for example
    input $(3, T)$ means that one trigger leading into {\tt AND}-gate
    number 3 is set to true (T). (Note that any trigger can be set only
    {\bf once}, see the section on simplifying assumptions above.)

\subsection{Set of states $S$} First, for any positive integer $m\in\mathbb{N}$,
    we define $$[n]^* := \{0,1,\ldots,n\} \cup\{-1\}$$ 
    and set $$S = [n_1]^* \times [n_2]^* \times \cdots \times [n_k]^*.$$
    So every state is a vector. We now give the interpretation of these
    state vectors: If $v\in S$, then if $v_i >0$ means that $v_i$ inputs
    have {\bf not yet} been set to TRUE.  If $v_i=1$, this means that all but $1$
    inputs are set to TRUE, and if we receive another TRUE input for {\tt AND}-gate $i$,
    then {\tt AND}-gate $i$ has all inputs set to TRUE and is TRUE itself,
    which also renders the output of the {\tt OR}-gate TRUE. If we receive a
    FALSE input into {\tt AND}-gate $i$, that is the member of the input alphabet
    is $(i, F)$, then immediately we know that the output of {\tt AND}-gate $i$ 
    must be FALSE. To indicate this, we set $v_i = -1$. Note that if 
    our state vector is $-1$ at every position, that is $v_i = -1$ for all $i$, 
    then the output of the {\tt OR}-gate is FALSE too, and we have reached 
    a terminal state.
    
    Let's consider two examples:
    \begin{itemize}
        \item $k = 2$, and $v = (3,1)$. This means we have $2$ {\tt AND}-gates 
        and in the first gate, all except $3$ entry points are TRUE,
        and for the second {\tt AND}-gate, all except $1$ entries
        have been set to TRUE.
        \item $k= 3$ and $v = (2, 1, -1)$. Similar as above, except 
        {\tt AND}-gate number 3 has seen one entry point being set to FALSE 
        (which makes the whole {\tt AND}-gate 3 output FALSE). 
    \end{itemize}
    
\subsection{Set of terminal states  $F$} $F\subseteq S$ is defined by
    $$F:= F_{\text{true}} \cup F_{\text{false}},$$ where 
    $$F_{\text{true}} = \big\{v\in S: \exists i\in\{1,\ldots,k\}
    (v_i=0)\big\},$$
    and 
    $$F_{\text{false}} = \big\{v\in S: \forall i\in\{1,\ldots k\}
    (v_i = -1)\big\} = \big \{(-1,-1,\ldots,-1)\big\}.$$
    Note that $F_{\text{false}}$ only has $1$ element, the constant -1 vector, 
    which is reached when every {\tt AND}-gate has seen some FALSE 
    input. A final state in $F_{\text{true}}$ is reached when 
    for some {\tt AND}-gate, all the inputs have been set to true.
    Every member of $S\setminus F$ is considered to be an ``undefined'' 
    or non-final state.
\subsection{Initial state $s_0\in S$} We set $$s_0 = (n_1, n_2, \ldots, n_k).$$
Recall that this means that for every {\tt AND}-gate, NO inputs have been 
set to TRUE, because we are at the beginning of the process.

\subsection{Transition function $\delta: S\times \Sigma \to S$}
And now we need to define the transition function $\delta: S \times \Sigma
\to S$, which is the ``clockwork'' of the whole finite state machine.
\begin{itemize}
    \item End state behaviour: Whenever $v\in F$ and $\sigma \in \Sigma$ 
    we set $\delta(v, \sigma) = v$.
    \item Suppose $\sigma = (i, T) \in \Sigma$ where $i\in\{1,\ldots,k\}$, 
    and $v\in S$.
    \begin{enumerate}
        \item If $v_i = -1$ we set $\delta(v, \sigma) = v$.
        \item If $v_i \geq 1$ we set $\delta(v,\sigma) = v'$ where $v'_i = v_i -1$ 
        and $v'_j = v_j$ for all $j\in \{1,\ldots,k\}\setminus\{i\}$. 
        \item Note that we do not have to cover the case $v_i = 0$ as this 
        is an end state, and we covered end states in the bullet point above. 
    \end{enumerate}
    \item Suppose $\sigma = (i, F) \in \Sigma$ where $i\in\{1,\ldots,k\}$, 
    and $v\in S$.
    \begin{enumerate}
        \item If $v_i = -1$ we set $\delta(v, \sigma) = v$.
        \item If $v_i \geq 1$ we set $\delta(v,\sigma) = v'$ where $v'_i = -1$ 
        and $v'_j = v_j$ for all $j\in \{1,\ldots,k\}\setminus\{i\}$. 
        \item Note that we do not have to cover the case $v_i = 0$ as this 
        is an end state, and we covered end states in the bullet point above. 
    \end{enumerate}
\end{itemize}
Size of state space: It is easy to see that for any $n\in\mathbb{N}$ we 
have $$\text{card}([n]^*) = \big|[n]^*\big| = n+2.$$
So $$\text{card}(S) = \text{card}\big([n_1]^* \times [n_2]^* 
\times \ldots \times [n_k]^* \big) =
\prod_{i=1}^k (n_i+2).$$

%--------------------------
\section{A DNF FSM WITH ``TRIGGER MEMORY''}

It turns out that in our application, the third of the simplifying 
assumptions does not hold. Recall that this was:
\begin{quote}
    (3) Per item to be analyzed, every trigger gets only set
        once to 0 or 1, there are no corrections or repetitions.
\end{quote}

So, if triggers can be set more than once, we need some kind 
of ``memory'' to store the information which trigger has been set
to what truth value, and which triggers are still undefined. (In the
AA standup, we referred to that as the ``idempotence property''.)

Not surprisingly the state space (set of possible states) will increase
considerably.

In this section, we build a finite state machine that stores the state
of the trigger and therefore is not changed in its state if 
a trigger is set more than once. 

We are using the same setting, 
variables, and indices as in section \ref{fsm1}.

Now we give a formal description of the FSM. Outlook: it will have 
\begin{itemize}
    \item a large state space $S$ unfortunately (but we warned of this before), and
    \item some good news: a very simple transition 
        function $\delta:S\times \Sigma \to S$!
\end{itemize}

\subsection{Input alphabet $\Sigma$} Let $N = \max\{n_1, \ldots, n_k\}$ 
be the maximum number of triggers that any {\tt AND}-gate has. 
We set $$\Sigma = \{1,\ldots, k\}\times\{1,\ldots,N\} \times \{0,1\}.$$
In the last set of the Cartesian product, we interpret $0$ as FALSE and $1$ 
as TRUE.

We interpret $(i,j,b)\in \Sigma$ as \begin{quote}
    ``in {\tt AND}-gate number $i$ set trigger number $j$ to $b$.''
\end{quote}
This begs the question, what happens if $j>n_i$, that is $j$ is larger than 
the number of triggers that {\tt AND}-gate number $i$ has? We set the transition
function $\delta:S\times \Sigma\to S$ to just do nothing in that case, that is to 
output the same state again. We'll come to that in a moment.

\subsection{State space $S$} 
Our basic units of information now are
\begin{itemize}
\item 0, corresponding to FALSE
\item 1, corresponding to TRUE
\item $u$, corresponding to ``undefined'' (meaning a trigger has 
    not yet been set).
\end{itemize}
Set $$S := \{0,1,u\}^{n_1} \times \{0,1,u\}^{n_2}\times \ldots \times \{0,1,u\}^{n_k}.$$
The set $S$ represents all possible configurations of the triggers for each 
{\tt AND}-gates. (Recall
that {\tt AND}-gate number $i$ has $n_i$ triggers feeding into it.) 

So $v\in S$ is 
a vector of $k$ elements, and for $i\in\{1,\ldots,k\}$, the element $v_i$ 
is itself a vector of $n_i$ elements, each which is either $0,1$, or $u$. 

{\em Example.} Let $k=3$ and $n_1 = 2, n_2 = 5, n_3 = 3$. So we 
have $3$ {\tt AND}-gates, the first with 2 triggers, the second with 5, 
and the last with 3 triggers. One example member of $S$ would be
$$v:=\big((1,u), (u,u,0,0,1), (u, u, 0)\big).$$
So for instance we have $(v_1)_2 = u$, as the second element of 
the first vector is $u$, and $(v_2)_3 = 0$. 

\subsection{Set of terminal states $F$} 
Based on the classification lemma \ref{classify} we can write
$F := F_{\text{false}}\cup F_{\text{true}}$ where
\begin{enumerate}
    \item $F_{\text{false}}$ consists of all members of $S$
        such that for all $i\in\{1,\ldots,k\}$ the vector
        $v_i$ contains some element that is $0$, and
    \item $F_{\text{true}}$ consists of all members of $S$
        such that there is $i\in\{1,\ldots k\}$ such that
        the vector $v_i$ is the constant-1-vector (more formally,
        $$(v_i)_j = 1 \text{ for all } j\in \{1,\ldots, n_i\}.$$
\end{enumerate}

\subsection{Initial state $s_0\in S$} We set $s_0$ to be the
vector $$\big(\underbrace{(u,\ldots, u)}_{\text{length }n_1},
\underbrace{(u,\ldots,u)}_{\text{length }n_2},\ldots 
\underbrace{(u,\ldots, u)}_{\text{length }n_k}\big).$$
Or more formally set $(v_i)_j = u$ for all $i\in\{1,\ldots, k\}$ and 
$j\in\{1,\ldots,n_i\}$ and set $s_0 :=v$.

\subsection{Transition function $\delta: S\times \Sigma \to S$}
\begin{itemize}
    \item End state behaviour: for $e\in F, \sigma\in \Sigma$ we set
        $\delta(e,\sigma) = e$.
    \item Suppose $v\in S \setminus F$ and let $\sigma = (i, j, b)\in \Sigma$, 
        where $i\in \{1,\ldots, k\}, j\in \{1,\ldots, N\}$
        \footnote{(recall $N = \max\{n_1,\ldots, n_k\}$)}, and $b\in \{0,1\}$. 
        We distinguish two cases:

        \begin{itemize}
            \item $j > n_i$: then do nothing, i.e. $\delta(v, (i,j,b)) = v$.
            \item $j \leq n_i$: set the value of $(v_i)_j$ to $b$, and otherwise
                leave $v$ unchanged. More formally, 
                define $v'\in S$ by:
                \begin{quote}
                $(v'_i)_j := b$, \\ 
                $(v'_i)_y := (v_i)_y$ for  $y \in \{1,\ldots n_i\} \setminus \{j\}$, and \\
                $v'_z = v_z$  for $z \in \{1,\ldots,k\}\setminus \{i\}.$
                \end{quote}

                \vspace*{2mm}

                Then set $\delta(v, (i,j,b)) := v'$.
        \end{itemize}
\end{itemize}

Note that this finite state machine not only has a trigger memory,
(so it has the idempotence property that we discussed in the AA 
meeting), it even allows for ``corrections'', that is
triggers that receive input TRUE first and then get
corrected to FALSE (this was not in the requirements). 

The state space size of this FSM is $$\text{card}(S) = 3^{(\sum_{i=1}^k n_i)}.$$

(Note that $\sum_{i=1}^k n_i$ is the total number of triggers.)

%--------------------------
\section{An ``order-theoretical'' approach}
% - - - - - - -
\subsection{Informal description of the idea} 
Suppose we are given a Boolean function
$$f: \{0,1\}^n\to\{0,1\}$$
that is entirely made up of $\land$ and $\lor$ and does not use 
negation ($\neg$). Assume that 
we are given the truth value (0 or 1) for some 
of the $n$ triggers, and for others, we do not know their truth
value. 
Then we propose the following procedure:
\begin{itemize}
\item Set all triggers with unknown truth value to 0 (FALSE) and 
calculate $f()$.
\item Set all triggers with unknown truth value to 1 (TRUE) and 
calculate $f()$.
\end{itemize}
If the output values of $f$ calculated above {\bf agree}, we can conclude 
(with an argument given in  the 
next section) that $f$ will give the same output {\bf no matter} what
truth values we give the triggers with unknown truth value. Therefore 
we can give an output with confidence in that case.

If the two values do not agree, we cannot give an output for $f$ given 
the partial information of the triggers of which we know the truth value.
% - - - - - - -
\subsection{(Order-theoretical) justification of the method given above}
\begin{definition} A {\em partially ordered set} or {\em poset} for short 
is a pair $(P,\leq_P)$ consisting of a base set $P$ and a binary relation 
$\leq_R \,  \subseteq \, P\times P$ such that 
\begin{enumerate}
    \item $(p,p)\in \,\leq_P$ for all $p\in P$ (``reflexivity'');
    \item $(p,q), (q,p) \in \, \leq_P$ implies $p=q$ (``anti-symmetry''), and 
    \item $(p,q), (q,r)\in \, \leq_P$ implies $(p,r)\in \, \leq_P$ (``transitivity'').
\end{enumerate}
Usually we write $p \leq_P q$ instead of $(p,q)\in\leq_P$.
\end{definition}

As an easy example of a poset, consider $\{0,1\}$ with the relation 
$$\leq = \{(0,0), (0,1), (1,1)\}.$$ For short, we say $\{0,1\}$ is 
{\em ordered by} $0\leq 1$.

The next example will give a hint of why these definitions will 
be important later: Let $P = \{0,1\}^n$ the set of all $\{0,1\}$-vectors
of length $n$, and we say that for $x,y\in \{0,1\}^n$ we have
$$x \leq y \text{ if and only if } 
x_i \leq y_i \text{ for all }i\in\{1,\ldots,n\}.$$

It is an easy exercise to verify that this gives us a poset.

\begin{definition} Let $(P, \leq_P), (Q, \leq_Q)$ be posets. 
A map $f:P\to Q$ is said to be {\em monotone} if
$$x\leq_P y \implies f(x)\leq_Q f(y) \text{ for all }x,y\in P.$$
\end{definition}

The following are examples of monotone functions:
\begin{itemize}
    \item $\land: \{0,1\}^2\to \{0,1\}$, 
    \item $\lor: \{0,1\}^2\to \{0,1\}$.
\end{itemize}

The next lemma deals with the composition of monotone functions and 
is quite easy to prove. 

\begin{lemma} If $P, Q, Z$ are posets and $f: P\to Q, g: Q\to Z$ are 
monotone functions, then so is $$g\circ f: P\to Z.$$
\end{lemma}

So in particular, we have:

\begin{corollary}\label{andor_lemma}
Any Boolean function $f:\{0,1\}\to \{0,1\}$ not containing negation ($\neg$),
but only consisting of $\land, \lor$ is a composition of the monotone 
functions $\lor$ and $\land$ (respectively, possibly multiple copies of these)
and it is therefore {\em monotone}.
\end{corollary}

We can apply the following ``sandwich lemma'' directly to our situation.

\begin{lemma}\label{sandwich_lemma} 
If $P, Q$ are posets and $f:P\to Q$ is monotone, and 
$a<b \in P$ such that $f(a) = f(b)$, then:
\begin{quote}
for all $x\in P$ with $a\leq x \leq b$ we have $f(a) = f(x) = f(b)$. 
\end{quote}
\end{lemma}

Let's revisit our starting situation. We are given a 
Boolean function $f:\{0,1\}\to \{0,1\}$ not containing negation ($\neg$),
but only consisting of $\land, \lor$. By lemma \ref{andor_lemma} we 
know that $f$ is monotone.

Now, we are ready to use the sandich lemma \ref{sandwich_lemma}. 
We know the truth values for some of the inputs. If we set all other 
inputs to 0, we get variable $a$ in the sandwich lemma: we 
go as low as possible. If we set all inputs with unknown truth value 
to $1$, we go as high as possible, that is, we get variable $b$ 
in the sandwich lemma. 

Any assignment of $\{0,1\}$ to the inputs (triggers) with unknown
truth value will result in an input vector $x$ that is {\bf sandwiched} 
between $a$ and $b$, i.e. $$a \leq x \leq b.$$ 

So we meet the pre-condition of the sandwich lemma and get that 
no matter what the input vector $x$ looks like, we get $f(x) = f(a) = f(b)$.

So we can give a definite output for $f$ even for our partial input.

\end{document}